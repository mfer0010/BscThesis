\chapter{Conclusions and Further Work}
\label{chp6}
\paragraph{ }In this study, \acs{LOF} and \acs{DBSCAN} were applied to data gathered from sensors around the moment of injection from the \acs{SPS} to the \acs{LHC} as unsupervised anomaly detection algorithms on various datasets. These anomalous points were then presented and it was concluded that the best performing algorithm in this case was \acs{LOF} when applied to the full dataset with all the provided parameters. Furthermore, the beam displacement over time was shown and the nature of the anomalies were presented in 3D plots.

\paragraph{ }Similar to Valentino \textit{et. al.}'s study of 2017 \cite{Valentino2017}, this proposed method can positively identify anomalous injections. However the method could use some tweaking as just like in Halilovic's thesis \cite{Halilovic2018}, the best performance still ''leaves something to be desired'' as there were still a large number of false positives being detected.

\paragraph{ }One limitation in this study came from the loss in performance of this algorithm on the \acs{PCA} Model. This could stem from the fact that when performing \acs{PCA}, only 80\% of the true variance of the data was kept. It would be interesting to see if the results would improve if more principal components were kept.

\paragraph{ }As anomaly detection in particle accelerators using unsupervised machine learning has not been explored much as of yet, further research in the area could prove to be beneficial not just for particle accelerator research but also for time series research in general. Further studies on properly fitting the parameters and their effect on the performance of the algorithms would be interesting. Also, as these techniques have the potential of detecting anomalies not being picked up by the \acs{IQC}, a study on implementing such a method in real time to work in conjunction with the \acs{IQC} could prove beneficial to reduce the number of anomalous injections in the \acs{LHC} and save researchers time and money when running these tests. Furthermore, now that a labelled dataset was created for this study, future work on using supervised learning on this data could result in some interesting results, possibly leading to a more accurate model.