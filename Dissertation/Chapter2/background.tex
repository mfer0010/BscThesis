\chapter{Background and Literature Review}
\section{Introduction}

\section{Understanding the Problem Domain}
%%Paragraph on why the LHC is used goes here
%% TO DO

%% How the LHC works:
%% TO DO

%%The Particular area of interest to this study:
\paragraph{ }The \acs{LHC} is filled to the required maximum intensity by injecting electron bunch trains from the \acs{SPS} through a transfer line using kicker magnets [CITE HERE]. This is a challenging task given the high energy of the beam, the very small apertures and the delivery precision's tight tolerances, thus multiple sensors are installed around the \acs{CERN} particle accelerator complex [CITE HERE] which gather readings and data that can be used to check the quality of the injected beam. For this particular study, the sensors around the injection from the \acs{SPS} to the \acs{LHC} will be of particular interest. This data is stored using \acs{CERN}'s \ac{LS} [CITE HERE]. While many studies have been made using this logged data and lots of statistical tests have been done with regards to injection quality checks for the \acs{LHC} (such as [CITE HERE] and [CITE HERE]), no literature was uncovered where researchers used unsupervised machine learning methods to analyse this data.

%% On the Current IQC software
\paragraph{ }The \ac{IQC} software currently installed has a set of hard-coded rules for detecting anomalies in the \acs{SPS}-\acs{LHC} injection [CITE HERE], however there are documented cases in the past where situations occurred which were outside the originally foreseen rules and were therefore not caught as anomalies.

\section{Feature Selection and Reduction Techniques}

%% Intro:
%% What are Feature selection and reduction techniques and how will these be used in this FYP (TO DO)

%% PCA:
%% TO DO
\paragraph{ } \acs{PCA} uses statistical and mathematical techniques to reduce the dimension of large data sets, thus allowing a large data set to be interpreted in less variables called principal components (\cite{Richardson2009}). This non-parametric method can be used as a means of revealing the simplified structures underlying complex datasets with minimal effort. The fact that this technique is non-parametric gives it the advantage that each result is unique and only dependent on the provided data set since no parameter tweaking is required (\cite{Shlens2014}) however, this is also a weakness of this technique as there is no way of exploiting prior expert knowledge on the data set.

%% Recursive Feature Selection:
%% TO DO

%% Multilinear Discriminant Analysis (MDA)
%% TO DO


\section{Unsupervised Anomaly Detection Techniques}