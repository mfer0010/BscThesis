\chapter{Background and Literature Review}
\section{Understanding the Problem Domain}
%% Beam Injection:
%% TO DO
\paragraph{ }The aim of the \ac{LHC} at \acs{CERN} is to accelerate and collide two proton beams \cite{Valentino2017}. In order to fill the \acs{LHC} with a beam of the required intensity, twelve injections consisting of a number of electron bunches of around 1 \acs{MJ} of stored energy each are required \cite{Drosdal2011}. This is a challenging task given the high energy of the beam, the very small apertures and the delivery precision's tight tolerances. Thus, multiple sensors are installed around the \acs{CERN} particle accelerator complex \cite{Lefevre2008} which gather readings and data that can be used to check the quality of the injected beam. 

%%The Particular area of interest to this study:
\paragraph{ }  For this particular study, data generated from the sensors around the injection from the \acs{SPS} to the \acs{LHC} will be of particular interest. This data is stored using \acs{CERN}'s \ac{LS} \cite{Roderick2013}. While many studies have been made using this logged data and lots of statistical tests have been done with regards to injection quality checks for the \acs{LHC} (such as \cite{Drosdal2011} and \cite{Kain2010}), no literature was uncovered where researchers used unsupervised machine learning methods to analyse this data. Figure \ref{fig::SPStoLHCInjection} highlights the particular area of interest of this study.


\begin{figure}[t]
	\centering
	\includegraphics[width=0.4\textwidth]{CERNComplex}
	\caption[The CERN Particle Accelerator Complex]{Diagram of the particular area of interest of the CERN Particle Accelerator Complex for this study}
	\label{fig::SPStoLHCInjection}
\end{figure}

%% On the Current IQC software
\paragraph{ }The \ac{IQC} software currently installed has a set of hard-coded rules for detecting anomalies in the \acs{SPS}-\acs{LHC} injection \cite{Drosdal2011}, however there are documented cases in the past where situations occurred which were outside the originally foreseen rules and were therefore not caught as anomalies.

\section{Feature Selection and Reduction Techniques}

%% Intro:
%% What are Feature selection and reduction techniques and how will these be used in this FYP (TO DO)

%% PCA:
\paragraph{ } \acs{PCA} uses statistical and mathematical techniques to reduce the dimension of large data sets, thus allowing a large data set to be interpreted in less variables called principal components \cite{Richardson2009}. This non-parametric method can be used as a means of revealing the simplified structures underlying complex datasets with minimal effort. The fact that this technique is non-parametric gives it the advantage that each result is unique and only dependent on the provided data set since no parameter tweaking is required \cite{Shlens2014} however, this is also a weakness of this technique as there is no way of exploiting prior expert knowledge on the data set.

%% Recursive Feature Selection:
%% TO DO

%% Multilinear Discriminant Analysis (MDA)
%% TO DO


\section{Unsupervised Anomaly Detection Techniques}