\begin{abstract}
Filling the LHC is a challenging task and anomalous injections are common when operating such a large machine. In this dissertation, unsupervised machine learning techniques for anomaly detection were used to analyse past data in order to understand the sources of the anomalies for improving the LHC machine availability and performance reach. Data from sensors around the moment of injection from the SPS to the LHC were analysed and different feature sets were chosen as input parameters for the anomaly detection algorithms. In this study, LOF was found to achieve the best performance when run on the full feature dataset with an accuracy of 92.76\% for Beam 1 and 91.08\% for Beam 2. The anomalous points are also visualised in 3D plots which serve to help researchers understand better the nature of these anomalies. Furthermore, the beam's position drift with time is also presented and is shown to also be a factor in impacting the injection quality.
\end{abstract}